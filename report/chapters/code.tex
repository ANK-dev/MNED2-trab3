\chapter{Código Computacional}

\subsection{Código Principal (\texttt{main.h} \& \texttt{main.c})}

\subsubsection{Arquivo \texttt{main.h}}

\begin{Verbatim}[fontsize=\footnotesize]
/******************************************************************************
 *        Métodos Numéricos para Equações Diferenciais II -- Trabalho 3       *
 *                          Ariel Nogueira Kovaljski                          *
 ******************************************************************************/

/*======================= Parâmetros a serem ajustados =======================*/

#define LX      (20.0)              /* comprimento do domínio (em m)          */
#define NX      (400)               /* número de células                      */
#define DELTA_X (LX/NX)             /* largura de cada célula (em m)          */
#define U_BAR   (1.0)               /* velocidade de escoamento (em m/s)      */
#define T_FINAL (2.0)               /* tempo final da simulação (em segundos) */
#define COURANT (0.8)               /* número de courant                      */

#define DELTA_T (COURANT*DELTA_X/U_BAR) /* passo de tempo (em segundos)       */

#define A (100.0)                   /*                                */
#define B (1.5)                     /*                                */
#define C (4.0)                     /* Parâmetros da condição inicial */
#define D (6.0)                     /*                                */
#define E (2.0)                     /*                                */

/*============================================================================*/

#define MIN(a,b) (((a) < (b)) ? (a) : (b))
#define MAX(a,b) (((a) > (b)) ? (a) : (b))
#define MAX3(a,b,c) (MAX(MAX((a),(b)), (c)))

/* Métodos utilizados para o cálculo de Q neste trabalho */
enum methods {UPWIND, SUPERBEE, VAN_ALBADA};

void listParameters();
void initializeArray(double arr[], int len, double a, double b, double c,
                     double d, double e);
double thetaPlusHalf(double arr[], int i);
double thetaMinusHalf(double arr[], int i);
void calculateQ(double old[], double new[], double (*psi)(double theta));
double upwind(double theta);
double superbee(double theta);
double vanAlbada(double theta);
void printAndSaveResults(double arr[], int len, int method);
\end{Verbatim}

\subsubsection{Arquivo \texttt{main.c}}
\begin{Verbatim}[fontsize=\footnotesize]
/******************************************************************************
 *        Métodos Numéricos para Equações Diferenciais II -- Trabalho 3       *
 *                          Ariel Nogueira Kovaljski                          *
 ******************************************************************************/

#include "main.h"

#include <stdio.h>
#include <stdlib.h>
#include <math.h>
#include <string.h>

int main(void)
{
    double Q_new[NX];   /* Array de Q no tempo n+1 */
    double Q_old[NX];   /* Array de Q no tempo n */

    puts("MNED II - Trabalho 3\n"
         "====================\n"
         "por Ariel Nogueira Kovaljski\n");

    listParameters();

    /* Cálculo de Q via método Upwind */
    puts("Calculando Q via metodo Upwind...");
    initializeArray(Q_old, NX, A, B, C, D, E);
    initializeArray(Q_new, NX, A, B, C, D, E);
    calculateQ(Q_old, Q_new, upwind);
    printAndSaveResults(Q_new, NX, UPWIND);

    /* Cálculo de Q via método Superbee */
    puts("Calculando Q via metodo Superbee...");
    initializeArray(Q_old, NX, A, B, C, D, E);
    initializeArray(Q_new, NX, A, B, C, D, E);
    calculateQ(Q_old, Q_new, superbee);
    printAndSaveResults(Q_new, NX, SUPERBEE);

    /* Cálculo de Q via método Van Albada */
    puts("Calculando Q via metodo Van Albada...");
    initializeArray(Q_old, NX, A, B, C, D, E);
    initializeArray(Q_new, NX, A, B, C, D, E);
    calculateQ(Q_old, Q_new, vanAlbada);
    printAndSaveResults(Q_new, NX, VAN_ALBADA);

    return 0;
}

void listParameters()
{
    puts("Parametros\n----------");
    puts("Constantes da equacao:");
    printf("DELTA_T = %f, DELTA_X = %f, U_BAR = %3.2e\n\n",
           DELTA_T, DELTA_T, U_BAR);
    puts("Constantes da simulacao:");
    printf("LX = %f, NX = %d, T_FINAL = %f\n\n", LX, NX, T_FINAL);
}

/* Inicializa um array para um valor de entrada */
void initializeArray(double arr[], int len, double a, double b, double c,
                     double d, double e)
{
    /*************************************************************
    *
    *                    Condição de contorno
    *                c(x,0) = exp(-A(x-B)²) + s(x)
    *
    *           onde s(x) = { E,    se  C <= x <= D
    *                       { 0,    c.c.
    */

    int i;
    double x, s;

    for (i = 0; i < len; ++i) {
        x = i * DELTA_X;
        s = (x >= c && x <= d ? e : 0);
        arr[i] = exp( -a * ((x - b)*(x - b)) ) + s;
    }
}

double thetaPlusHalf(double arr[], int i)
{
    int a, b, c;

    /* Aplica condição de contorno / volume fantasma */
    a = (i - 1) < 0 ? 0 : (i - 1);
    b = i;
    c = i + 1;

    /* Workaround para divisão por zero */
    if (arr[c] - arr[b] == 0) {
        return (arr[b] - arr[a]) / 1e-10;
    }

    return (arr[b] - arr[a]) / (arr[c] - arr[b]);
}

double thetaMinusHalf(double arr[], int i)
{
    int a, b, c;

    /* Aplica condição de contorno / volume fantasma */
    a = (i - 2) < 0 ? 0 : (i - 2);
    b = (i - 1) < 0 ? 0 : (i - 1);
    c = i;

    /* Workaround para divisão por zero */
    if (arr[c] - arr[b] == 0) {
        return (arr[b] - arr[a]) / 1e-10;
    }

    return (arr[b] - arr[a]) / (arr[c] - arr[b]);
}

/* calcula Q, recebendo como ponteiro a função `psi` */
void calculateQ( double old[], double new[], double (*psi)(double
theta) )
{
    int i;
    double t = 0;

    do {
        /*
         *                     Fronteira esquerda
         *                |=@=|=@=|=@=|=@=|=@=|=@=|=@=|
         *                  ^
         */
        i = 0;
        new[i] = old[i] - COURANT/2 * (1-COURANT) * (
                     psi(thetaPlusHalf(old, i)) * (old[i+1] - old[i])
                 );

        /*
         *                            Centro
         *                |=@=|=@=|=@=|=@=|=@=|=@=|=@=|
         *                      ^   ^   ^   ^   ^
         */
        for (i = 1; i < NX - 1; ++i) {
            new[i] = old[i] - COURANT * (
                         old[i] - old[i-1]
                     ) - COURANT/2 * (1-COURANT) * (
                           psi(thetaPlusHalf(old, i))  * (old[i+1] - old[i]  )
                         - psi(thetaMinusHalf(old, i)) * (old[i]   - old[i-1])
                     );
        }

        /*
         *                      Fronteira direita
         *                |=@=|=@=|=@=|=@=|=@=|=@=|=@=|
         *                                          ^
         */
        new[i] = old[i] - COURANT * (
                     old[i] - old[i-1]
                 ) - COURANT/2 * (1-COURANT) * (
                     - psi(thetaMinusHalf(old, i)) * (old[i] - old[i-1])
                 );

        /* Atualiza array de valores antigos com os novos para o próximo
           passo de tempo */
        for (i = 0; i < NX; ++i) {
            old[i] = new[i];
        }

    } while ( (t += DELTA_T) <= T_FINAL);
}

/* Método Upwind */
double upwind(double theta)
{
    return (theta = 0);
}

/* Método Superbee */
double superbee(double theta)
{
    return MAX3(0, MIN(1, 2*theta), MIN(2, theta));
}

/* Método Van Albada */
double vanAlbada(double theta)
{
    return ( (theta*theta) + theta ) / ( (theta*theta) + 1 );
}

/* Imprime na tela e salva os resultados num arquivo de saída */
void printAndSaveResults(double arr[], int len, int method)
{
    int i;
    char filename[50], file0[100], file1[100];
    FILE *results_file;     /* Ponteiro para o arquivo de resultados */

    /* Prepara nome do arquivo de saída */
    switch (method) {
        case UPWIND:
            sprintf(filename, "%s%d%s", "results", method, ".txt");
            break;
        case SUPERBEE:
            sprintf(filename, "%s%d%s", "results", method, ".txt");
            break;
        case VAN_ALBADA:
            sprintf(filename, "%s%d%s", "results", method, ".txt");
            break;
    }

    /* Imprime os resultados no console */
    printf("Q[%d] (tempo final: %.2fs) = [", NX, T_FINAL);
    for (i = 0; i < len - 1; ++i) {
        printf("%f, ", arr[i]);
    }
    printf("%f]\n\n", arr[i]);

    /* Error Handling -- Verifica se é possível criar/escrever o
                         arquivo de resultados */
    sprintf(file0, "%s%s", "./results/",    filename);
    sprintf(file1, "%s%s", "./../results/", filename);
    if (   ( results_file = fopen(file0, "w") ) == NULL
        && ( results_file = fopen(file1, "w") ) == NULL ) {
        fprintf(stderr,
                "[ERR] Houve um erro ao escrever o arquivo \"%s\"! "
                "Os resultados nao foram salvos.\n", filename);
        exit(1);
    }

    /* Adiciona os resultados no arquivo "results.txt" */
    fprintf(results_file,
            "nx=%d\n"
            "Delta_t=%f\n"
            "Delta_x=%f\n"
            "t_final=%f\n"
            "u_bar=%f\n",
            NX, DELTA_T, DELTA_X, T_FINAL, U_BAR);
    fputs("********************\n", results_file);

    for (i = 0; i < len; ++i) {
        fprintf(results_file, "%f,%f\n", i * DELTA_X, arr[i]);
    }

    fclose(results_file);   /* Fecha o arquivo */

    printf("[INFO] Os resultados foram salvos no arquivo \"%s\" "
    "no diretorio \"results/\".\n\n", filename);
}
\end{Verbatim}

\pagebreak
\subsection{Código do Gráfico (\texttt{plot\_graph.py})}

\begin{Verbatim}[fontsize=\footnotesize]
################################################################################
#         Métodos Numéricos para Equações Diferenciais II -- Trabalho 3        #
#                           Ariel Nogueira Kovaljski                           #
################################################################################

import matplotlib.pyplot as plt


x0, y0 = [], []
x1, y1 = [], []
x2, y2 = [], []
parameters = {}

def main():
    # Abre arquivos para leitura
    # Arquivo Upwind
    try:
        f0 = open('./results/results0.txt', 'r')
    except OSError as err:
        print("Erro ao abrir o arquivo 'results0.txt':", err)

    # Arquivo Superbee
    try:
        f1 = open('./results/results1.txt', 'r')
    except OSError as err:
        print("Erro ao abrir o arquivo 'results1.txt':", err)

    # Arquivo Van Albada
    try:
        f2 = open('./results/results2.txt', 'r')
    except OSError as err:
        print("Erro ao abrir o arquivo 'results2.txt':", err)


    # Extrai informações dos arquivos
    data_extract(f0, x0, y0)
    data_extract(f1, x1, y1)
    data_extract(f2, x2, y2)

    # Plota os gráficos de cada método individualmente
    plot_graph(x0, y0, 0)
    plot_graph(x1, y1, 1)
    plot_graph(x2, y2, 2)

    # Plota todos os gráficos na mesma janela
    plot_all_graphs()


def data_extract(f, x, y):
    for line_number, line in enumerate(f):
        if (line_number + 1) < 6:
            # Adiciona parâmetros da simulação em um dicionário
            parameters[line.split('=')[0]] = (line.split('=')[1]).split('\n')[0]
        elif (line_number + 1) == 6:
            # Pula linha separadora
            pass
        else:
            # Separa valores na lista `x` e na lista `y`
            x.append( float(line.split(',')[0]) )
            y.append( float((line.split(',')[1]).split('\n')[0]) )


def plot_graph(x, y, id):
    methods = {0: 'Upwind', 1:'Superbee', 2:'Van Albada'}
    colors  = {0: 'red',    1: 'green',   2: 'blue'}
    fig,ax  = plt.subplots()
    fig.set_size_inches(12, 7)
    ax.grid(True)
    ax.set(ylim=(0, 2.2))

    plt.suptitle(f"Método {methods[id]}: Concentração X Posição")
    plt.title(rf"$nx = {parameters['nx']}$, "
              rf"$\Delta t = {parameters['Delta_t']}$, "
              rf"$\Delta x = {parameters['Delta_x']}$, "
              rf"$t_{{final}} = {parameters['t_final']}$, "
              rf"$\bar{{u}} = {parameters['u_bar']}$, ", fontsize=8)

    plt.xlabel("posição x (m)")
    plt.ylabel("concentração Q")
    plt.plot(x,y,'ko-', markerfacecolor=colors[id], markeredgecolor='k')
    plt.draw()

def plot_all_graphs():
    methods = {0: 'Upwind', 1:'Superbee', 2:'Van Albada'}
    colors  = {0: 'red',    1: 'green',   2: 'blue'}
    fig,ax  = plt.subplots()
    fig.set_size_inches(12, 7)
    ax.grid(True)
    ax.set(ylim=(0, 2.2))

    plt.suptitle(f"Todos os métodos: Concentração X Posição")
    plt.title(rf"$nx = {parameters['nx']}$, "
              rf"$\Delta t = {parameters['Delta_t']}$, "
              rf"$\Delta x = {parameters['Delta_x']}$, "
              rf"$t_{{final}} = {parameters['t_final']}$, "
              rf"$\bar{{u}} = {parameters['u_bar']}$, ", fontsize=8)
    plt.xlabel("posição x (m)")
    plt.ylabel("concentração Q")

    plt.plot(x0, y0, f'{colors[0][0]}-', markerfacecolor=colors[0],
             label=methods[0])
    plt.plot(x1, y1, f'{colors[1][0]}-', markerfacecolor=colors[1],
             label=methods[1])
    plt.plot(x2, y2, f'{colors[2][0]}-', markerfacecolor=colors[2],
             label=methods[2])
    plt.legend(loc='upper right')
    plt.savefig(f"./results/fig/methods_t={parameters['t_final']}.png")
    plt.show()

if __name__ == "__main__":
    main()
\end{Verbatim}