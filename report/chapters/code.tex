\chapter{Código Computacional}

\subsection{Código Principal (\texttt{main.h} \& \texttt{main.c})}

\subsubsection{Arquivo \texttt{main.h}}

\begin{Verbatim}[fontsize=\footnotesize]
/******************************************************************************
 *        Métodos Numéricos para Equações Diferenciais II -- Trabalho 2       *
 *                          Ariel Nogueira Kovaljski                          *
 ******************************************************************************/

/*======================= Parâmetros a serem ajustados =======================*/

#define LX      (10.0)              /* comprimento do domínio (em m)          */
#define NX      (200)               /* número de células                      */
#define DELTA_X (LX/NX)             /* largura de cada célula (em m)          */
#define U_BAR   (2.0)               /* velocidade de escoamento (em m/s)      */
#define T_FINAL (1.0)               /* tempo final da simulação (em segundos) */
#define COURANT (0.9)               /* C: número de courant                   */

#define DELTA_T (COURANT*DELTA_X/U_BAR) /* passo de tempo (em segundos)       */

#define A (100.0)                   /*                                */
#define B (1.5)                     /*                                */
#define C (4.0)                     /* Parâmetros da condição inicial */
#define D (6.0)                     /*                                */
#define E (2.0)                     /*                                */

/*============================================================================*/

/* Métodos utilizados para o cálculo de Q neste trabalho */
enum methods {FTBS, LF, LW, BW};

void listParameters();
void initializeArray(double arr[], int len, double a, double b, double c,
                     double d, double e);
void calculateQ_FTBS(double old_arr[], double new_arr[]);
void calculateQ_LF(double old_arr[], double new_arr[]);
void calculateQ_LW(double old_arr[], double new_arr[]);
void calculateQ_BW(double old_arr[], double new_arr[]);
void printAndSaveResults(double arr[], int len, int method);
\end{Verbatim}

\subsubsection{Arquivo \texttt{main.c}}
\begin{Verbatim}[fontsize=\footnotesize]
/******************************************************************************
 *        Métodos Numéricos para Equações Diferenciais II -- Trabalho 2       *
 *                          Ariel Nogueira Kovaljski                          *
 ******************************************************************************/

#include <stdio.h>
#include <stdlib.h>
#include <math.h>
#include <string.h>
#include "main.h"

int main(void)
{
    double Q_new[NX];   /* Array de Q no tempo n+1 */
    double Q_old[NX];   /* Array de Q no tempo n */

    puts("MNED II - Trabalho 2\n"
         "====================\n"
         "por Ariel Nogueira Kovaljski\n");

    listParameters();

    /* Cálculo de Q via método Forward Time-Backward Space (FTBS) */
    initializeArray(Q_old, NX, A, B, C, D, E);
    initializeArray(Q_new, NX, A, B, C, D, E);
    calculateQ_FTBS(Q_old, Q_new);
    printAndSaveResults(Q_new, NX, FTBS);

    /* Cálculo de Q via método Lax-Friedrichs */
    initializeArray(Q_old, NX, A, B, C, D, E);
    initializeArray(Q_new, NX, A, B, C, D, E);
    calculateQ_LF(Q_old, Q_new);
    printAndSaveResults(Q_new, NX, LF);

    /* Cálculo de Q via método Lax-Wendroff */
    initializeArray(Q_old, NX, A, B, C, D, E);
    initializeArray(Q_new, NX, A, B, C, D, E);
    calculateQ_LW(Q_old, Q_new);
    printAndSaveResults(Q_new, NX, LW);

    /* Cálculo de Q via método Beam-Warmimg */
    initializeArray(Q_old, NX, A, B, C, D, E);
    initializeArray(Q_new, NX, A, B, C, D, E);
    calculateQ_BW(Q_old, Q_new);
    printAndSaveResults(Q_new, NX, BW);

    return 0;
}

void listParameters()
{
    puts("Parametros\n----------");
    puts("Constantes da equacao:");
    printf("DELTA_T = %f, DELTA_X = %f, U_BAR = %3.2e\n\n",
           DELTA_T, DELTA_T, U_BAR);
    puts("Constantes da simulacao:");
    printf("NX = %d, T_FINAL = %f\n\n", NX, T_FINAL);
}

/* Inicializa um array para um valor de entrada */
void initializeArray(double arr[], int len, double a, double b, double c,
                     double d, double e)
{
    /*************************************************************
    *
    *                    Condição de contorno
    *                c(x,0) = exp(-A(x-B)²) + s(x)
    *
    *           onde s(x) = { E,    se  C <= x <= D
    *                       { 0,    c.c.
    */

    int i;
    double x, s;

    for (i = 0; i < len; ++i) {
        x = i * DELTA_X;
        s = (x >= c && x <= d ? e : 0);
        arr[i] = exp( -a * ((x - b)*(x - b)) ) + s;
    }
}

/* Calcula as concentrações na malha ao longo do tempo */
/* Método Forward Time-Backward Space (FTBS) */
void calculateQ_FTBS(double old[], double new[])
{
    int i;
    int progress = 0, progress_count = 0;
    int progress_incr = (T_FINAL/DELTA_T) * 5.0 / 100;
    double t = 0;

    puts("Calculando FTBS...");

    do {
        /*************************************************************
        *
        *            |==@==|==@==|==@==|==@==|==@==|==@==|
        *               ^
        *            Para o volume da fronteira esquerda
        *        o índice 0 refere-se ao volume nº 1 da malha
        */
        new[0] = old[0];

        /*************************************************************
        *
        *            |==@==|==@==|==@==|==@==|==@==|==@==|
        *                     ^     ^     ^     ^     ^
        *                  Para os volumes do centro
        *               e na fronteira direita da malha
        */
        for (i = 1; i < NX; ++i) {
            new[i] = old[i] - U_BAR*DELTA_T/DELTA_X * (
                old[i] - old[i-1]
            );
        }

        /* Incrementa o progresso a cada 5% */
        if (progress_count == progress_incr){
            progress_count = 0;
            ++progress;
            printf("\rCalculando... %d%% concluido", progress * 5);
            fflush(stdout);
        }

        /* Atualiza array de valores antigos com os novos para o próximo
        passo de tempo */
        for (i = 0; i < NX; ++i) {
            old[i] = new[i];
        }

        /* Incrementa contador para cada 5% */
        ++progress_count;

    } while ( (t += DELTA_T) <= T_FINAL );
}

/* Método Lax-Friedrichs */
void calculateQ_LF(double old[], double new[])
{
    int i;
    int progress = 0, progress_count = 0;
    int progress_incr = (T_FINAL/DELTA_T) * 5.0 / 100;
    double t = 0;

    puts("Calculando Lax-Friedrichs...");

    do {
        /*************************************************************
        *
        *            |==@==|==@==|==@==|==@==|==@==|==@==|
        *               ^
        *            Para o volume da fronteira esquerda
        *        o índice 0 refere-se ao volume nº 1 da malha
        */
        new[0] = (
                    old[1] + old[0]
                 )/2 - U_BAR*DELTA_T/(2*DELTA_X) * (
                    old[1] - old[0]
                 );

        /*************************************************************
        *
        *            |==@==|==@==|==@==|==@==|==@==|==@==|
        *                     ^     ^     ^     ^
        *             Para os volumes do centro da malha
        */
        for (i = 1; i < NX - 1; ++i) {
            new[i] = (
                        old[i+1] + old[i-1]
                     )/2 - U_BAR*DELTA_T/(2*DELTA_X) * (
                        old[i+1] - old[i-1]
                     );
        }

        /*************************************************************
        *
        *            |==@==|==@==|==@==|==@==|==@==|==@==|
        *                                             ^
        *             Para o volume da fronteira direita
        *            i possui valor de NX - 1 nesse ponto
        */
        new[i] = (
                    old[i] + old[i-1]
                 )/2 - U_BAR*DELTA_T/(2*DELTA_X) * (
                    old[i] - old[i-1]
                 );

        /* Incrementa o progresso a cada 5% */
        if (progress_count == progress_incr){
            progress_count = 0;
            ++progress;
            printf("\rCalculando... %d%% concluido", progress * 5);
            fflush(stdout);
        }

        /* Atualiza array de valores antigos com os novos para o próximo
        passo de tempo */
        for (i = 0; i < NX; ++i) {
            old[i] = new[i];
        }

        /* Incrementa contador para cada 5% */
        ++progress_count;

    } while ( (t += DELTA_T) <= T_FINAL );
}

/* Método Lax-Wendroff */
void calculateQ_LW(double old[], double new[])
{
    int i;
    int progress = 0, progress_count = 0;
    int progress_incr = (T_FINAL/DELTA_T) * 5.0 / 100;
    double t = 0;

    puts("Calculando Lax-Wendroff...");

    do {
        /*************************************************************
        *
        *            |==@==|==@==|==@==|==@==|==@==|==@==|
        *               ^
        *            Para o volume da fronteira esquerda
        *        o índice 0 refere-se ao volume nº 1 da malha
        */
        new[0] = old[0] - U_BAR*DELTA_T/(2*DELTA_X) * (
                    old[1] - old[0]
                 ) + (U_BAR*U_BAR)*(DELTA_T*DELTA_T)/(2*DELTA_X*DELTA_X) * (
                    old[1] - old[0]
                 );

        /*************************************************************
        *
        *            |==@==|==@==|==@==|==@==|==@==|==@==|
        *                     ^     ^     ^     ^
        *             Para os volumes do centro da malha
        */
        for (i = 1; i < NX - 1; ++i) {
            new[i] = old[i] - U_BAR*DELTA_T/(2*DELTA_X) * (
                old[i+1] - old[i-1]
            ) + (U_BAR*U_BAR)*(DELTA_T*DELTA_T)/(2*DELTA_X*DELTA_X) * (
                old[i+1] - 2*old[i] + old[i-1]
            );
        }

        /*************************************************************
        *
        *            |==@==|==@==|==@==|==@==|==@==|==@==|
        *                                             ^
        *             Para o volume da fronteira direita
        *            i possui valor de NX - 1 nesse ponto
        */
        new[i] = old[i] - U_BAR*DELTA_T/(2*DELTA_X) * (
            old[i] - old[i-1]
        ) + (U_BAR*U_BAR)*(DELTA_T*DELTA_T)/(2*DELTA_X*DELTA_X) * (
            old[i-1] - old[i]
        );

        /* Incrementa o progresso a cada 5% */
        if (progress_count == progress_incr){
            progress_count = 0;
            ++progress;
            printf("\rCalculando... %d%% concluido", progress * 5);
            fflush(stdout);
        }

        /* Atualiza array de valores antigos com os novos para o próximo
        passo de tempo */
        for (i = 0; i < NX; ++i) {
            old[i] = new[i];
        }

        /* Incrementa contador para cada 5% */
        ++progress_count;

    } while ( (t += DELTA_T) <= T_FINAL );
}

/* Método Beam-Warming */
void calculateQ_BW(double old[], double new[])
{
    int i;
    int progress = 0, progress_count = 0;
    int progress_incr = (T_FINAL/DELTA_T) * 5.0 / 100;
    double t = 0;

    puts("Calculando Beam-Warming...");

    do {
        /*************************************************************
        *
        *            |==@==|==@==|==@==|==@==|==@==|==@==|
        *               ^
        *            Para o volume da fronteira esquerda
        *        o índice 0 refere-se ao volume nº 1 da malha.
        *     O método de Lax-Wendroff é utilizado para este caso
        */
        new[0] = old[0] - U_BAR*DELTA_T/(2*DELTA_X) * (
                    old[1] - old[0]
                 ) + (U_BAR*U_BAR)*(DELTA_T*DELTA_T)/(2*DELTA_X*DELTA_X) * (
                    old[1] - old[0]
                 );

        /*************************************************************
        *
        *            |==@==|==@==|==@==|==@==|==@==|==@==|
        *                     ^
        *         Para o volume logo após a fronteira esquerda
        */
        new[1] = old[1] - 3 * U_BAR*DELTA_T/(2*DELTA_X) * (
                    old[1] - old[0]
                 ) + (U_BAR*U_BAR)*(DELTA_T*DELTA_T)/(2*DELTA_X*DELTA_X) * (
                    old[1] - old[0]
                 );

        /*************************************************************
        *
        *            |==@==|==@==|==@==|==@==|==@==|==@==|
        *                           ^     ^     ^     ^
        *                  Para os volumes do centro
        *               e na fronteira direita da malha
        */
        for (i = 2; i < NX; ++i) {
            new[i] = old[i] - U_BAR*DELTA_T/(2*DELTA_X) * (
                        3*old[i] - 4*old[i-1] + old[i-2]
                     ) + (U_BAR*U_BAR)*(DELTA_T*DELTA_T)/(2*DELTA_X*DELTA_X) * (
                        old[i] - 2*old[i-1] + old[i-2]
                     );
        }

        /* Incrementa o progresso a cada 5% */
        if (progress_count == progress_incr){
            progress_count = 0;
            ++progress;
            printf("\rCalculando... %d%% concluido", progress * 5);
            fflush(stdout);
        }

        /* Atualiza array de valores antigos com os novos para o próximo
        passo de tempo */
        for (i = 0; i < NX; ++i) {
            old[i] = new[i];
        }

        /* Incrementa contador para cada 5% */
        ++progress_count;

    } while ( (t += DELTA_T) <= T_FINAL );
}

/* Imprime na tela e salva os resultados num arquivo de saída */
void printAndSaveResults(double arr[], int len, int method)
{
    int i;
    char filename[50], file0[50], file1[50];
    FILE *results_file;     /* Ponteiro para o arquivo de resultados */

    /* Prepara nome do arquivo de saída */
    switch (method) {
        case FTBS:
            snprintf(filename, 50, "%s%d%s", "results", method, ".txt");
            break;
        case LF:
            snprintf(filename, 50, "%s%d%s", "results", method, ".txt");
            break;
        case LW:
            snprintf(filename, 50, "%s%d%s", "results", method, ".txt");
            break;
        case BW:
            snprintf(filename, 50, "%s%d%s", "results", method, ".txt");
            break;
    }

    /* Imprime os resultados no console */
    printf("\n\nQ[%d] (tempo final: %.2fs) = [", NX, T_FINAL);
    for (i = 0; i < len - 1; ++i) {
        printf("%f, ", arr[i]);
    }
    printf("%f]\n\n", arr[i]);

    /* Error Handling -- Verifica se é possível criar/escrever o arquivo de
                         resultados */
    snprintf(file0, 50, "%s%s", "./results/", filename);
    snprintf(file1, 50, "%s%s", "./../results/", filename);
    if (   ( results_file = fopen(file0, "w") ) == NULL
        && ( results_file = fopen(file1, "w") ) == NULL) {
        fprintf(stderr, "[ERR] Houve um erro ao escrever o arquivo \"%s\"! "
                        "Os resultados nao foram salvos.\n", filename);
        exit(1);
    }

    /* Adiciona os resultados no arquivo "results.txt" */
    fprintf(results_file,
            "nx=%d\n"
            "Delta_t=%f\n"
            "Delta_x=%f\n"
            "t_final=%f\n"
            "u_bar=%f\n",
            NX, DELTA_T, DELTA_X, T_FINAL, U_BAR);
    fputs("********************\n", results_file);

    for (i = 0; i < len; ++i) {
        fprintf(results_file, "%f,%f\n", i * DELTA_X, arr[i]);
    }

    fclose(results_file);   /* Fecha o arquivo */

    printf("[INFO] Os resultados foram salvos no arquivo \"%s\" "
           "no diretorio \"results/\".\n\n", filename);
}
\end{Verbatim}

\subsection{Código do Gráfico (\texttt{plot\_graph.py})}

\begin{Verbatim}[fontsize=\footnotesize]
################################################################################
#         Métodos Numéricos para Equações Diferenciais II -- Trabalho 2        #
#                           Ariel Nogueira Kovaljski                           #
################################################################################

import matplotlib.pyplot as plt


x0, y0 = [], []
x1, y1 = [], []
x2, y2 = [], []
x3, y3 = [], []
parameters = {}

def main():
    # Abre arquivos para leitura
    # Arquivo FTBS
    try:
        f0 = open('./results/results0.txt', 'r')
    except OSError as err:
        print("Erro ao abrir o arquivo 'results0.txt':", err)

    # Arquivo Lax-Friedrichs
    try:
        f1 = open('./results/results1.txt', 'r')
    except OSError as err:
        print("Erro ao abrir o arquivo 'results1.txt':", err)

    # Arquivo Lax-Wendroff
    try:
        f2 = open('./results/results2.txt', 'r')
    except OSError as err:
        print("Erro ao abrir o arquivo 'results2.txt':", err)

    # Arquivo Beam-Warming
    try:
        f3 = open('./results/results3.txt', 'r')
    except OSError as err:
        print("Erro ao abrir o arquivo 'results3.txt':", err)

    # Extrai informações dos arquivos
    data_extract(f0, x0, y0)
    data_extract(f1, x1, y1)
    data_extract(f2, x2, y2)
    data_extract(f3, x3, y3)

    # Plota os gráficos de cada método
    plot_graph(x0, y0, 0)
    plot_graph(x1, y1, 1)
    plot_graph(x2, y2, 2)
    plot_graph(x3, y3, 3)

    # Exibe os gráficos
    plt.show()


def data_extract(f, x, y):
    for line_number, line in enumerate(f):
        if (line_number + 1) < 6:
            # Adiciona parâmetros da simulação em um dicionário
            parameters[line.split('=')[0]] = (line.split('=')[1]).split('\n')[0]
        elif (line_number + 1) == 6:
            # Pula linha separadora
            pass
        else:
            # Separa valores na lista `x` e na lista `y`
            x.append( float(line.split(',')[0]) )
            y.append( float((line.split(',')[1]).split('\n')[0]) )


def plot_graph(x, y, id):
    methods = {0:'FTBS', 1:'Lax-Friedrichs', 2:'Lax-Wendroff', 3:'Beam-Warming'}
    colors  = {0: 'red', 1: 'green',         2: 'blue',        3: 'yellow'}
    fig,ax  = plt.subplots()
    fig.set_size_inches(8, 7)
    ax.grid(True)

    plt.suptitle(f"Método {methods[id]}: Concentração X Posição")
    plt.title(rf"$nx = {parameters['NX']}$, "
              rf"$\Delta t = {parameters['DELTA_T']}$, "
              rf"$\Delta x = {parameters['DELTA_X']}$, "
              rf"$t_{{final}} = {parameters['T_FINAL']}$, "
              rf"$\bar{{u}} = {parameters['U_BAR']}$, ", fontsize=8)

    plt.xlabel("posição x (m)")
    plt.ylabel("concentração Q")
    plt.plot(x,y,'ko-', markerfacecolor=colors[id], markeredgecolor='k')

if __name__ == "__main__":
    main()
\end{Verbatim}