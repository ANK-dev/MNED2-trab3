\chapter{Discussão}

Observa-se que a aplicação de diferentes métodos numéricos para um mesmo
problema --- neste caso, a EDP da advecção --- resulta em soluções bastante
diferentes. Métodos de primeira ordem como FTBS e Lax-Friedrichs, possuem erro
de truncamento na ordem de $\frac{\partial^2 c}{\partial x^2}$ o que introduz
uma pequena difusão numérica e oscilações. Métodos de segunda ordem como
Lax-Wendroff e Beam-Warming apesar de tecnicamente melhores, apresentam erro de
truncamento na ordem de $\frac{\partial^3 c}{\partial x^3}$ o que resulta em
dispersão numérica e maiores oscilações.

Em relação aos refinamentos da malha, nota-se que todos os métodos se comportam
de maneira similar. Uma diminuição no número de nós $nx$ resulta em uma curva
mais ``pontiaguda'' e um aumento deste resulta em uma curva mais ``suave''.
Como consequência, pode-se concluir que o refinamento da malha aproxima os
gráficos --- pertencentes ao domínio discreto --- da solução contínua, dada
pela EDP analítica.

Em relação às mudanças em $t_{\text{final}}$, observa-se que novamente os
métodos apresentam comportamento similar. Conforme o avanço do tempo, o efeito
advectivo faz com que as concentrações se desloquem para a direita, o que está
de acordo com a velocidade $\bar{u} > 0$. Dado um tempo suficientemente longo,
as concentrações se estabilizam em zero.