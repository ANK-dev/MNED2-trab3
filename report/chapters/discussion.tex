\chapter{Discussão}

Observa-se que a aplicação de diferentes métodos numéricos para um mesmo
problema --- neste caso, a EDP da advecção --- resulta em soluções bastante
diferentes. Métodos de primeira ordem como FTBS, possuem erro
de truncamento na ordem de $\frac{\partial^2 c}{\partial x^2}$ o que introduz
um grau mais elevado de difusão numérica. Métodos TVD, possuem as vantagens dos
Métodos de Alta Resolução --- baixa difusão numérica, devido ao erro de
truncamento na ordem de $\frac{\partial^3 c}{\partial x^3}$ --- sem os seus
defeitos --- oscilações espúrias.

Em relação aos refinamentos da malha, nota-se que todos os métodos se comportam
de maneira similar. Uma diminuição no número de nós $nx$, prejudica a acurácia
da solução, pois a baixa resolução esconde as variações abruptas nas curvas. O
aumento de $nx$ apresenta um resultado mais fiel ao real, porém demanda mais
poder computacional para o cálculo da malha.

Em relação às mudanças em $t_{\text{final}}$, observa-se que conforme o avanço
do tempo, o efeito advectivo faz com que as concentrações se desloquem para a
direita, o que está de acordo com a velocidade $\bar{u} > 0$. Erros causados
pela difusão numérica mostram que métodos de primeira ordem como o FTBS não são
os mais adequados para esta análise. Dentre os métodos TVD, o que apresentou
melhor resultado foi o Superbee, pois manteve-se mais próximo da curva
analítica comparado com o Van Albada, mesmo para valores de $t_{\text{final}}$
elevados.