\chapter{Metodologia}
Neste capítulo serão abordados os passos e métodos utilizados para se
obter a solução numérica do problema proposto.

\section{Condições Inicial e de Contorno}
A resolução de qualquer equação diferencial parcial (EDP) requer a
determinação de sua condição(ões) inicial(ais) e de contorno. Como
proposto pelo trabalho, a concentração inicial da malha é dada pela
seguinte equação
\begin{equation}
    c(x,0) = e^{-A(x - B)} + s(x)
\end{equation}
e as condições de contorno são dadas pelas seguintes equações

% Equações lado-a-lado
\medskip
\noindent
\begin{minipage}{.5\linewidth}
    \begin{equation}
        \left( \frac{\partial c}{\partial x} \right)_{x=0}^t = 0
    \end{equation}
\end{minipage}%
\begin{minipage}{.5\linewidth}
    \begin{equation}
        \left( \frac{\partial c}{\partial x} \right)_{x=L_x}^t = 0
    \end{equation}
\end{minipage}

\medskip
\noindent para o contorno esquerdo e direito, respectivamente.

Usando o conceitos de volumes fantasmas, é possível determinar o valor dos
volumes no contorno, através das seguintes aproximações, para o contorno
esquerdo,

\[
    \left(\frac{\partial c}{\partial c}\right)_{x=0}^t
    \approx
    \frac{Q_1^n - Q_0^n}{\Delta x} = 0
\]
\begin{equation}
    \therefore\ Q_1^n = Q_0^n
\end{equation}
e para o contorno direito,
\[
    \left(\frac{\partial c}{\partial c}\right)_{x=L_x}^t
    \approx
    \frac{Q_{nx+1}^n - Q_{nx}^n}{\Delta x} = 0
\]
\begin{equation}
    \therefore\ Q_{nx+1}^n = Q_{nx}^n
\end{equation}

\section{Métodos Numéricos}
Os métodos numéricos utilizados para a resolução da equação de advecção são
diversos. Nesta seção serão tratados os três métodos utilizados neste trabalho.

Como base para todos os métodos, será utilizada a equação \ref{vol. fin.},
modificada ligeiramente, conforme descrito na proposta do trabalho. Esta versão
modificada pode ser vista abaixo,
\begin{equation}
\begin{split}
    Q_i^{n+1} = {} & Q_i^n - C \left( Q_i^n - Q_{i-1}^n \right) \\
                   & - \frac{1}{2}C(1-C)
    \left[
        \psi \left( \theta_{i+1/2}^n \right) \left( Q_{i+1/2}^n-Q_i^n \right)
        - \psi \left( \theta_{i-1/2}^n \right) \left(Q_i^n - Q_{i-1}^n \right)
    \right]
\end{split}
\end{equation}
onde C é o número de Courant, $\theta_{i\pm1/2}^n$ são funções adicionais,
definidas como,
\[
    \theta_{i-1/2}^n = \frac{Q_{i-1}^n - Q_{i-2}^n}{Q_i^n - Q_{i-1}^n}
    \qquad
    \text{e}
    \qquad
    \theta_{i+1/2}^n = \frac{Q_i^n - Q_{i-1}^n}{Q_{i+1}^n - Q_i^n}
\]
e $\psi$ é um dos métodos numéricos a serem adotados.

\subsection{Métodos \textit{Upwind}}
Problemas hiperbólicos, como a equação da advecção, possuem informação (ondas)
que se propagam com uma velocidade e sentido característico. A utilização de
métodos \textit{upwind} leva em conta essa característica, permitindo uma
modelagem mais acurada do fenômeno tratado. O método \textit{upwind} escolhido
para a resolução da equação da advecção é o \textit{forward time-backward space}
(FTBS).

\subsubsection{Forward Time-Backward Space (FTBS)}
O FTBS trata da ideia de que, para a equação da advecção unidimensional, há
apenas uma única onda que se propaga. O método \textit{upwind} determina o
valor de $Q_i^{n+1}$, onde, para um $\bar{u} > 0$, resulta em um fluxo da
esquerda para a direita, de forma que a concentração de cada volume $Q_i^{n+1}$
depende dos volumes atual $Q_i^n$ e anterior $Q_{i-1}^n$. Sua forma como função
$\psi$ é:
\begin{equation}\label{FTBS}
    \psi(\theta) = 0
\end{equation}

\subsection{Métodos TVD}
A utilização de métodos \textit{upwind} de primeira ordem, apesar de simples,
acaba por introduzir significativa difusão numérica, impactando negativamente a
acurácia da solução. Os métodos TVD introduzem um termo corretivo, de maneira a
minimizar a influência da difusão numérica sobre o resultado final. Os métodos
TVD escolhidos para a resolução da equação da advecção são o Superbee e Van
Albada.

\subsubsection{Superbee}
Para o Superbee, sua forma como função $\psi$ é:
\begin{equation}\label{Superbee}
    \psi(\theta) = \max(0, \min(1,2\theta), \min(2,\theta))
\end{equation}

\subsubsection{Van Albada}
Para Van Albada, sua forma como função $\psi$ é:
\begin{equation}\label{Van Albada}
    \psi(\theta) = \frac{\theta^2 + \theta}{\theta^2 + 1}
\end{equation}

\section{Estabilidade}
Se trata do comportamento do algoritmo e seus valores numéricos frente aos
parâmetros de entrada. Um algoritmo estável se comporta de maneira esperada
frente a uma faixa específica de valores de entrada.

A partir da condição de estabilidade para a equação da advecção-difusão,
utilizada no primeiro trabalho,
\begin{equation}
    \Delta t
    \leq
    C\left(
        \frac{1}{\frac{2\alpha}{{\Delta x}^2} + \frac{\bar{u}}{\Delta x}}
    \right)
\end{equation}
considerando que não há componente difusivo $\alpha$ na equação de advecção, o
mesmo pode ser igualado a 0.
\[
    \Delta t
    \leq
    C\left(
        \frac{1}{\frac{2(0)}{{\Delta x}^2} + \frac{\bar{u}}{\Delta x}}
    \right)
\]
\[
    \Delta t \leq C\left( \frac{1}{\frac{\bar{u}}{\Delta x}} \right)
\]
\begin{equation}
    \therefore\ \Delta t \leq C\frac{\Delta x}{\bar{u}}
\end{equation}
obtém-se assim a condição de estabilidade para os métodos numéricos para a
equação da advecção, onde $C$ trata-se do número de Courant: adota-se um valor
de 0,8 para estre trabalho, visando satisfazer a condição de estabilidade e
evitar possíveis erros numéricos durante o cálculo computacional. O método
obtido trata-se de um método \emph{condicionalmente estável}, ou seja que
depende de uma faixa de valores para garantir a estabilidade de seu
funcionamento.

\section{Programação}
Aliado destes conceitos, foi possível construir um programa em linguagem C que
calcula as concentrações para cada célula ao longo do tempo, para cada método
aqui citado, exportando arquivos de texto com os resultados; estes arquivos,
então, são lidos por um \textit{script} Python que gera os gráficos
correspondentes.

O programa principal possui a seguinte estrutura, descrita em C:

\begin{Verbatim}[fontsize=\footnotesize]
// Vetores para concentração no tempo `n' e no tempo `n+1', respectivamente
double Q_old[];
double Q_new[];

// Para cada método, os vetores são inicializados e as concentrações são
// calculadas

// Método FTBS
// Inicializa ambos os vetores com a função de concentração inicial
initializeArray(Q_old, NX, A, B, C, D, E)
initializeArray(Q_new, NX, A, B, C, D, E)

// Calcula Q através do método FTBS (upwind)
calculateQ(Q_old, Q_new, upwind);

// Imprime na tela e salva os resultados no arquivo de texto
printAndSaveResults(Q_new, NX, UPWIND);

// Método Superbee
initializeArray(Q_old, NX, A, B, C, D, E)
initializeArray(Q_new, NX, A, B, C, D, E)
calculateQ(Q_old, Q_new, superbee);
printAndSaveResults(Q_new, NX, SUPERBEE);

// Método Van Albada
initializeArray(Q_old, NX, A, B, C, D, E)
initializeArray(Q_new, NX, A, B, C, D, E)
calculateQ(Q_old, Q_new, vanAlbada);
printAndSaveResults(Q_new, NX, VAN_ALBADA);
\end{Verbatim}

\noindent A função \verb|initializeArray| possui a seguinte estrutura:
\begin{Verbatim}[fontsize=\footnotesize]
int i;
double x, s;

// Laço `for' percorre todos os índices do vetor
for (i = 0; i < nx; ++i) {

    // Posição `x' (m) é definida como índice do volume * largura do volume
    x = i * Delta_x;

    // `s' recebe o valor de `E' somente se C <= x <= D
    s = (x >= C && x <= D ? E : 0);

    // Cada índice do vetor `Q' é inicializado segundo a condição inicial
    Q[i] = exp( -A * ((x - B)*(x - B)) ) + s;
}
\end{Verbatim}

\noindent Cada função \verb|calculateQ| possui a seguinte estrutura:
\begin{Verbatim}[fontsize=\footnotesize]
// Cálculo de Q, iterado ao longo do tempo
int i;
double t = 0;

do {
    // Cálculo do volume da fronteira esquerda
    i = 0;
    new[i] = old[i] - COURANT/2 * (1-COURANT) * (
                psi(thetaPlusHalf(old, i)) * (old[i+1] - old[i])
             );

    // Cálculo dos volumes do centro da malha
    for (i = 1; i < NX - 1; ++i) {
        new[i] = old[i] - COURANT * (
                     old[i] - old[i-1]
                 ) - COURANT/2 * (1-COURANT) * (
                       psi(thetaPlusHalf(old, i))  * (old[i+1] - old[i]  )
                     - psi(thetaMinusHalf(old, i)) * (old[i]   - old[i-1])
                );
    }

    // Cálculo do volume da fronteira direita
    new[i] = old[i] - COURANT * (
                 old[i] - old[i-1]
             ) - COURANT/2 * (1-COURANT) * (
                 - psi(thetaMinusHalf(old, i)) * (old[i] - old[i-1])
             );

    // Atualiza array de valores antigos com os novos para o próximo
    // passo de tempo
    for (i = 0; i < NX; ++i) {
        old[i] = new[i];
    }

// Incrementa passo de tempo
} while ( (t += DELTA_T) <= T_FINAL);
\end{Verbatim}

\noindent Cada função \verb|thetaMinusHalf| / \verb|thetaPlusHalf| possui a
seguinte estrutura:
\begin{Verbatim}[fontsize=\footnotesize]
int a, b, c;

// Aplica condição de contorno / volume fantasma
if (method == "thetaMinusHalf") {
    a = (i - 2) < 0 ? 0 : (i - 2);
    b = (i - 1) < 0 ? 0 : (i - 1);
    c = i;
} else if (method == "thetaPlusHalf") {
    a = (i - 1) < 0 ? 0 : (i - 1);
    b = i;
    c = i + 1;
}

// Workaround para divisão por zero
if (arr[c] - arr[b] == 0) {
    return (arr[b] - arr[a]) / 1e-10;
    /* return 0; */
}

return (arr[b] - arr[a]) / (arr[c] - arr[b]);
\end{Verbatim}

São definidos dois vetores, \verb|Q_old[]| e \verb|Q_new[]|, que correspondem as
concentrações $Q$ no tempo $n$ e $n+1$, respectivamente. Antes do cálculo das
concentrações, os vetores são inicializados, em um simples laço \verb|for|,
seguindo a função de concentração inicial $e^{-A(x-b)^2} + s(x)$.

Ao ser chamada, a função \verb|calculateQ| recebe um ponteiro para a função
\verb|psi| correspondente. Isto permite o reuso de código, pois os passos do
cálculo de $Q$ são idênticos para todas as funções.

A cada iteração do laço \verb|do-while|, o tempo \verb|t| é incrementado por uma
quantidade \verb|DELTA_T|, que obedece as regras de estabilidade descritas na
seção anterior. Ao longo da iteração, o vetor \verb|Q_new[]| é calculado para as
fronteiras e para o centro da malha, em função de \verb|Q_old[]|. Para o
cálculo de cada função \verb|psi|, antes é calculado o valor de
$\theta_{i\pm1/2}^n$ por suas respectivas funções \verb|thetaMinusHalf| e
\verb|thetaPlusHalf|.

O cálculo de $\theta_{i\pm1/2}^n$ aplica as condições de contorno, evitando
assim o acesso de índices fora do domínio do vetor. Para alguns casos onde o
denominador resultaria em zero, é aplicado um \textit{workaround}, isto é, uma
medida mitigadora, onde o denominador é definido como um número pequeno
$\sim 10^{-10}$.

Antes do fim da iteração, os vetores \verb|Q_old[]| são atualizados com os
valores de \verb|Q_new[]|, o tempo é incrementado, e então a nova iteração é
iniciada.

Ao fim da execução, o vetor \verb|Q_new[]|, terá os resultados da concentração
de cada volume da malha, correspondente a cada índice do vetor, no tempo
\verb|t = t_final|. Os pares índice-concentração são exportados em um arquivo
de texto, para cada método, linha-a-linha, para serem lidos e plotados pelo
\textit{script} Python.
