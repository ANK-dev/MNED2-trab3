\chapter{Conclusão}
Equações diferenciais são uma ferramenta poderosa para a modelagem de problemas
físicos. A EDP da advecção, muito utilizada na modelagem do escoamento de
fluídos, pode ser difícil ou até impossível de resolver analiticamente.

Desta forma, se faz necessário o uso de métodos numéricos, os quais permitem a
obtenção de uma solução aproximada para o problema. Diferentes métodos possuem
diferentes comportamentos, com vantagens e desvantagens para cada situação.
Neste trabalho, foi estudado o comportamento de quatro métodos, na busca da
solução da equação da advecção.

Seguindo as condições de estabilidade, foi possível perceber que métodos como o
FTBS e Lax-Friedrichs possuem um comportamento bastante suave, com poucas
oscilações, mas não tão boa acurácia. Métodos de alta resolução como
Lax-Wendroff e Beam-Warming possuem melhor acurácia, mas sofrem de oscilações
espúrias que prejudicam sua precisão.

A escolha dentre diferentes métodos é necessária dependendo dos parâmetros do
problema físico e do projeto de engenharia a ser estudado. É necessário
estabelecer um \textit{tradeoff} entre tempo de projeto vs.\ acurácia da
solução, de maneira a manter os custos sob controle.